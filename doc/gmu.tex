\documentclass[11pt,a4paper]{article}
\usepackage[czech]{babel}
\usepackage[utf8]{inputenc}
\usepackage{times}
\usepackage{url}
\usepackage[textwidth=15.2cm,textheight=23cm]{geometry}
\usepackage{xcolor}

\usepackage{graphicx}

%\usepackage{fancyvrb}
%\DefineVerbatimEnvironment{verbatim}{Verbatim}{}

\usepackage[bf]{caption}

\usepackage[hyperindex,
  plainpages=false,
  pdftex,
  colorlinks,
  pdfborder={0 0 0},
  pdfpagelabels]{hyperref}

\pdfcompresslevel=9

\newcommand{\myincludegraphics}[4]{
  \begin{figure}[!h]
  \centering
  \includegraphics[#1]{#2}
  \caption{#3.} \label{#4}
  \end{figure}
}

% titulní stránka a obsah
\newcommand{\titlepageandcontents}{
  % credits for template go to: Martin Striz
\begin{titlepage}

\vspace*{1cm}

\begin{figure}
  \centering
  \includegraphics[height=6cm]{img/fit.pdf}
\end{figure}

\vspace*{5mm}

\begin{center}
\begin{Large}
Projekt do předmětu GMU -- Grafické a multimediální procesory
\end{Large}
\end{center}

\vspace*{5mm}

\begin{center}
\begin{Huge}
Fyzikální simulace pevného tělesa na GPU \\
\end{Huge}
\end{center}

\vspace*{1cm}

\begin{center}
\begin{Large}
\today
\end{Large}
\end{center}

\vfill

\begin{flushleft}
\begin{large}
\begin{tabular}{ll}

\bf Řešitel:\hspace{3mm} & Ondřej Janošík (\verb_xjanos12@stud.fit.vutbr.cz_) \\
& Fakulta Informačních Technologií \\
& Vysoké Učení Technické v~Brně

\end{tabular}
\end{large}
\end{flushleft}

\end{titlepage}

% vim:set ft=tex expandtab enc=utf8:


  \pagestyle{plain}
  \pagenumbering{roman}
  \setcounter{page}{1}
  %\tableofcontents

  \newpage
  \pagestyle{plain}
  \pagenumbering{arabic}
  \setcounter{page}{1}
}

\def\uv#1{\iflanguage{english}{``#1''}%
                              {\quotedblbase #1\textquotedblleft}}%

% vim:set ft=tex expandtab enc=utf8:


\begin{document}
\titlepageandcontents

%---------------------------------------------------------------------------
\section{Zadání}

Zde napište informace k zadání (nejde jen o přepis toho, co je na webu;
komentujte vaše vlastní zpřesnění zadání, zaměření, důrazy, pojetí atd.). Text
strukturujte, použijte odrážky, číslování$\ldots$

Rozsah: cca 10 odrážek

%---------------------------------------------------------------------------
\section{Použité technologie}

Zde vypište, jaké technologie vaše řešení používá – co potřebuje k běhu, co
jste použili při tvorbě, atd. Text strukturujte, použijte odrážky,
číslování$\ldots$

Rozsah: cca 7 odrážek

%---------------------------------------------------------------------------
\section{Použité zdroje}

Zde vypište, které zdroje jste použili k tvorbě: hotový kód, hotová data
(obrázky, modely, $\ldots$), studijní materiály. Pokud vyplyne, že v projektu
je použit kód nebo data, která nejsou uvedena tady, jedná se o závažný problém
a projekt bude pravděpodobně hodnocen 0 body.

Rozsah: potřebný počet odrážek

%---------------------------------------------------------------------------
\section{Nejdůležitější dosažené výsledky}

Popište 3 věci, které jsou na vašem projektu nejlepší. Nejlépe ukažte a
komentujte obrázky, v nejhorším případě vypište textově.

%---------------------------------------------------------------------------
\section{Ovládání vytvořeného programu}

Stručně popište, jak se program ovládá (nejlépe odrážky rozdělené do
kategorií). Pokud se ovládání odchyluje od zkratek a způsobů obvykle
používaných v okýnkových nadstavbách operačních systémů, zdůvodněte, proč se
tak děje.

Rozsah: potřebný počet odrážek

%---------------------------------------------------------------------------
\section{Zvláštní použité znalosti}

Uveďte informace, které byly potřeba nad rámec výuky probírané na FIT.
Vysvětlete je pomocí obrázků, schémat, vzorců apod.

Rozsah: podle potřeby

%---------------------------------------------------------------------------
\section{Rozdělení práce v týmu}

\begin{itemize}
\item Franta: udělal tohle, udělal tableto, ještě taky toto, vedl tým.
\item Pepa: pracoval na tom, na tomhle a ještě na tomto.
\item Mařenka: vytvořila tohle, tamto a ještě něco.
\end{itemize}
Pokud to bude vhodné, použijte odrážky místo souvislých vět.

Rozsah: co nejstručnější tak, aby bylo zřejmé, jak byla dělena práce a za co v
projektu je kdo zodpovědný.

%---------------------------------------------------------------------------
\section{Co bylo nejpracnější}

Popište, co vám při řešení nejvíce komplikovalo život, s čím jste se museli
potýkat, co zabralo čas.

Rozsah: 5-10 řádků

%---------------------------------------------------------------------------
\section{Zkušenosti získané řešením projektu}

Popište, co jste se řešením projektu naučili. Zahrňte dovednosti obecně
programátorské, věci z oblasti počítačové grafiky, ale i spolupráci v týmu,
hospodaření s časem, atd.

Rozsah: formulujte stručně, uchopte cca 3-5 věcí

%---------------------------------------------------------------------------
\section{Autoevaluace}

Ohodnoťte vaše řešení v jednotlivých kategoriích (0 – nic neuděláno,
zoufalství, 100\% – dokonalost sama). Projekt, který ve finále obdrží plný
počet bodů, může mít složky hodnocené i hodně nízko. Uvedení hodnot blízkých
100\% ve všech nebo mnoha kategoriích může ukazovat na nepochopení problematiky
nebo na snahu kamuflovat slabé stránky projektu. Bodově hodnocena bude i
schopnost vnímat silné a slabé stránky svého řešení.

\paragraph{Technický návrh (50\%):} (analýza, dekompozice problému, volba
vhodných prostředků, $\ldots$)
Stručně (1-2 řádky) komentujte hodnocení.

\paragraph{Programování (50\%):} (kvalita a čitelnost kódu, spolehlivost běhu,
obecnost řešení, znovupoužitelnost, $\ldots$)
Stručně (1-2 řádky) komentujte hodnocení.

\paragraph{Vzhled vytvořeného řešení (50\%):} (uvěřitelnost zobrazení,
estetická kvalita, vhled GUI, $\ldots$)
Stručně (1-2 řádky) komentujte hodnocení.

\paragraph{Využití zdrojů (50\%):} (využití existujícího kódu a dat, využití
literatury, $\ldots$)
Stručně (1-2 řádky) komentujte hodnocení.

\paragraph{Hospodaření s časem (50\%):} (rovnoměrné dotažení částí projektu,
míra spěchu, chybějící části řešení, $\ldots$)
Stručně (1-2 řádky) komentujte hodnocení.

\paragraph{Spolupráce v týmu (50\%):} (komunikace, dodržování dohod, vzájemné
spolehnutí, rovnoměrnost, $\ldots$)
Stručně (1-2 řádky) komentujte hodnocení.

\paragraph{Celkový dojem (50\%):} (pracnost, získané dovednosti, užitečnost,
volba zadání, cokoliv, $\ldots$)
Stručně (5-10 řádků) komentujte hodnocení.

%---------------------------------------------------------------------------
\section{Doporučení pro budoucí zadávání projektů}

Co vám vyhovovalo a co nevyhovovalo na organizaci projektů? Které prvky by měly
být zachovány, zesíleny, potlačeny, eliminovány?

%---------------------------------------------------------------------------
\section{Různé}

Ještě něco by v dokumentaci mělo být? Napište to sem! Podle potřeby i založte
novou kapitolu.

\end{document}
% vim:set ft=tex expandtab enc=utf8:
